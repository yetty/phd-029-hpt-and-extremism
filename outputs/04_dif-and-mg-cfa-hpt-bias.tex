% Options for packages loaded elsewhere
\PassOptionsToPackage{unicode}{hyperref}
\PassOptionsToPackage{hyphens}{url}
%
\documentclass[
  10pt,
]{article}
\usepackage{amsmath,amssymb}
\usepackage{iftex}
\ifPDFTeX
  \usepackage[T1]{fontenc}
  \usepackage[utf8]{inputenc}
  \usepackage{textcomp} % provide euro and other symbols
\else % if luatex or xetex
  \usepackage{unicode-math} % this also loads fontspec
  \defaultfontfeatures{Scale=MatchLowercase}
  \defaultfontfeatures[\rmfamily]{Ligatures=TeX,Scale=1}
\fi
\usepackage{lmodern}
\ifPDFTeX\else
  % xetex/luatex font selection
\fi
% Use upquote if available, for straight quotes in verbatim environments
\IfFileExists{upquote.sty}{\usepackage{upquote}}{}
\IfFileExists{microtype.sty}{% use microtype if available
  \usepackage[]{microtype}
  \UseMicrotypeSet[protrusion]{basicmath} % disable protrusion for tt fonts
}{}
\makeatletter
\@ifundefined{KOMAClassName}{% if non-KOMA class
  \IfFileExists{parskip.sty}{%
    \usepackage{parskip}
  }{% else
    \setlength{\parindent}{0pt}
    \setlength{\parskip}{6pt plus 2pt minus 1pt}}
}{% if KOMA class
  \KOMAoptions{parskip=half}}
\makeatother
\usepackage{xcolor}
\usepackage[landscape,margin=0.7in]{geometry}
\usepackage{color}
\usepackage{fancyvrb}
\newcommand{\VerbBar}{|}
\newcommand{\VERB}{\Verb[commandchars=\\\{\}]}
\DefineVerbatimEnvironment{Highlighting}{Verbatim}{commandchars=\\\{\}}
% Add ',fontsize=\small' for more characters per line
\usepackage{framed}
\definecolor{shadecolor}{RGB}{248,248,248}
\newenvironment{Shaded}{\begin{snugshade}}{\end{snugshade}}
\newcommand{\AlertTok}[1]{\textcolor[rgb]{0.94,0.16,0.16}{#1}}
\newcommand{\AnnotationTok}[1]{\textcolor[rgb]{0.56,0.35,0.01}{\textbf{\textit{#1}}}}
\newcommand{\AttributeTok}[1]{\textcolor[rgb]{0.13,0.29,0.53}{#1}}
\newcommand{\BaseNTok}[1]{\textcolor[rgb]{0.00,0.00,0.81}{#1}}
\newcommand{\BuiltInTok}[1]{#1}
\newcommand{\CharTok}[1]{\textcolor[rgb]{0.31,0.60,0.02}{#1}}
\newcommand{\CommentTok}[1]{\textcolor[rgb]{0.56,0.35,0.01}{\textit{#1}}}
\newcommand{\CommentVarTok}[1]{\textcolor[rgb]{0.56,0.35,0.01}{\textbf{\textit{#1}}}}
\newcommand{\ConstantTok}[1]{\textcolor[rgb]{0.56,0.35,0.01}{#1}}
\newcommand{\ControlFlowTok}[1]{\textcolor[rgb]{0.13,0.29,0.53}{\textbf{#1}}}
\newcommand{\DataTypeTok}[1]{\textcolor[rgb]{0.13,0.29,0.53}{#1}}
\newcommand{\DecValTok}[1]{\textcolor[rgb]{0.00,0.00,0.81}{#1}}
\newcommand{\DocumentationTok}[1]{\textcolor[rgb]{0.56,0.35,0.01}{\textbf{\textit{#1}}}}
\newcommand{\ErrorTok}[1]{\textcolor[rgb]{0.64,0.00,0.00}{\textbf{#1}}}
\newcommand{\ExtensionTok}[1]{#1}
\newcommand{\FloatTok}[1]{\textcolor[rgb]{0.00,0.00,0.81}{#1}}
\newcommand{\FunctionTok}[1]{\textcolor[rgb]{0.13,0.29,0.53}{\textbf{#1}}}
\newcommand{\ImportTok}[1]{#1}
\newcommand{\InformationTok}[1]{\textcolor[rgb]{0.56,0.35,0.01}{\textbf{\textit{#1}}}}
\newcommand{\KeywordTok}[1]{\textcolor[rgb]{0.13,0.29,0.53}{\textbf{#1}}}
\newcommand{\NormalTok}[1]{#1}
\newcommand{\OperatorTok}[1]{\textcolor[rgb]{0.81,0.36,0.00}{\textbf{#1}}}
\newcommand{\OtherTok}[1]{\textcolor[rgb]{0.56,0.35,0.01}{#1}}
\newcommand{\PreprocessorTok}[1]{\textcolor[rgb]{0.56,0.35,0.01}{\textit{#1}}}
\newcommand{\RegionMarkerTok}[1]{#1}
\newcommand{\SpecialCharTok}[1]{\textcolor[rgb]{0.81,0.36,0.00}{\textbf{#1}}}
\newcommand{\SpecialStringTok}[1]{\textcolor[rgb]{0.31,0.60,0.02}{#1}}
\newcommand{\StringTok}[1]{\textcolor[rgb]{0.31,0.60,0.02}{#1}}
\newcommand{\VariableTok}[1]{\textcolor[rgb]{0.00,0.00,0.00}{#1}}
\newcommand{\VerbatimStringTok}[1]{\textcolor[rgb]{0.31,0.60,0.02}{#1}}
\newcommand{\WarningTok}[1]{\textcolor[rgb]{0.56,0.35,0.01}{\textbf{\textit{#1}}}}
\usepackage{longtable,booktabs,array}
\usepackage{calc} % for calculating minipage widths
% Correct order of tables after \paragraph or \subparagraph
\usepackage{etoolbox}
\makeatletter
\patchcmd\longtable{\par}{\if@noskipsec\mbox{}\fi\par}{}{}
\makeatother
% Allow footnotes in longtable head/foot
\IfFileExists{footnotehyper.sty}{\usepackage{footnotehyper}}{\usepackage{footnote}}
\makesavenoteenv{longtable}
\usepackage{graphicx}
\makeatletter
\def\maxwidth{\ifdim\Gin@nat@width>\linewidth\linewidth\else\Gin@nat@width\fi}
\def\maxheight{\ifdim\Gin@nat@height>\textheight\textheight\else\Gin@nat@height\fi}
\makeatother
% Scale images if necessary, so that they will not overflow the page
% margins by default, and it is still possible to overwrite the defaults
% using explicit options in \includegraphics[width, height, ...]{}
\setkeys{Gin}{width=\maxwidth,height=\maxheight,keepaspectratio}
% Set default figure placement to htbp
\makeatletter
\def\fps@figure{htbp}
\makeatother
\setlength{\emergencystretch}{3em} % prevent overfull lines
\providecommand{\tightlist}{%
  \setlength{\itemsep}{0pt}\setlength{\parskip}{0pt}}
\setcounter{secnumdepth}{5}
\usepackage{booktabs}
\usepackage{longtable}
\usepackage{array}
\usepackage{multirow}
\usepackage{float}
\usepackage{colortbl}
\usepackage{pdflscape}
\usepackage{threeparttable}
\usepackage{threeparttablex}
\usepackage[normalem]{ulem}
\usepackage{makecell}
\setlength{\LTcapwidth}{\textwidth}
\ifLuaTeX
  \usepackage{selnolig}  % disable illegal ligatures
\fi
\IfFileExists{bookmark.sty}{\usepackage{bookmark}}{\usepackage{hyperref}}
\IfFileExists{xurl.sty}{\usepackage{xurl}}{} % add URL line breaks if available
\urlstyle{same}
\hypersetup{
  pdftitle={Measurement checks --- HPT (Czech data)},
  pdfauthor={HPT and Extremism project},
  hidelinks,
  pdfcreator={LaTeX via pandoc}}

\title{Measurement checks --- HPT (Czech data)}
\usepackage{etoolbox}
\makeatletter
\providecommand{\subtitle}[1]{% add subtitle to \maketitle
  \apptocmd{\@title}{\par {\large #1 \par}}{}{}
}
\makeatother
\subtitle{DIF \& multi-group CFA by ideological attitude (ideological
contamination tests)}
\author{HPT and Extremism project}
\date{2025-12-12}

\begin{document}
\maketitle

{
\setcounter{tocdepth}{3}
\tableofcontents
}
\hypertarget{purpose-plan}{%
\section{Purpose \& plan}\label{purpose-plan}}

We assess whether \emph{ideological attitude groups} respond differently
to the HPT items \textbf{even at equal underlying HPT competence}.
Concretely:

\begin{itemize}
\tightlist
\item
  Create \textbf{low/high ideology} groups from FR-LF-mini and KSA-3
  (see codebook).\\
\item
  Run \textbf{item-level DIF} (ordinal logistic) for each HPT item.
\item
  Run \textbf{multi-group CFA} (configural → metric → scalar).
\item
  \textbf{Interpretation:} If we find pervasive DIF and/or scalar
  non-invariance, this supports the PCI RR H1 that HPT scores can be
  ideologically contaminated.
\end{itemize}

\hypertarget{setup}{%
\section{Setup}\label{setup}}

\begin{Shaded}
\begin{Highlighting}[]
\FunctionTok{options}\NormalTok{(}\AttributeTok{width =} \DecValTok{120}\NormalTok{)}
\FunctionTok{library}\NormalTok{(dplyr)}
\FunctionTok{library}\NormalTok{(tidyr)}
\FunctionTok{library}\NormalTok{(ggplot2)}
\FunctionTok{library}\NormalTok{(psych)}
\FunctionTok{library}\NormalTok{(knitr)}
\FunctionTok{library}\NormalTok{(stringr)}
\FunctionTok{library}\NormalTok{(janitor)}
\FunctionTok{library}\NormalTok{(difR)        }\CommentTok{\# DIF for ordinal items}
\FunctionTok{library}\NormalTok{(lavaan)      }\CommentTok{\# CFA / invariance}
\FunctionTok{library}\NormalTok{(semTools)    }\CommentTok{\# helpers}
\FunctionTok{library}\NormalTok{(car)         }\CommentTok{\# recode}
\FunctionTok{library}\NormalTok{(mirt)}
\end{Highlighting}
\end{Shaded}

\hypertarget{data}{%
\section{Data}\label{data}}

\begin{Shaded}
\begin{Highlighting}[]
\CommentTok{\# Load the dataset created in 00\_data{-}preparation}
\FunctionTok{load}\NormalTok{(}\StringTok{"normalised\_responses.RData"}\NormalTok{)}
\FunctionTok{stopifnot}\NormalTok{(}\FunctionTok{exists}\NormalTok{(}\StringTok{"normalised\_responses"}\NormalTok{))}
\NormalTok{dat\_raw }\OtherTok{\textless{}{-}}\NormalTok{ normalised\_responses}

\CommentTok{\# Cluster vars}
\NormalTok{dat\_raw }\OtherTok{\textless{}{-}}\NormalTok{ dat\_raw }\SpecialCharTok{\%\textgreater{}\%}
  \FunctionTok{mutate}\NormalTok{(}
    \AttributeTok{school\_id   =} \FunctionTok{as.factor}\NormalTok{(school\_id),}
    \AttributeTok{class\_label =} \FunctionTok{as.factor}\NormalTok{(class\_label),}
    \AttributeTok{class\_id    =} \FunctionTok{interaction}\NormalTok{(school\_id, class\_label, }\AttributeTok{drop =} \ConstantTok{TRUE}\NormalTok{)}
\NormalTok{  )}

\CommentTok{\# Coerce HPT items to numeric early (1{-}4 expected in codebook)}
\NormalTok{hpt\_items }\OtherTok{\textless{}{-}} \FunctionTok{c}\NormalTok{(}\FunctionTok{paste0}\NormalTok{(}\StringTok{"POP"}\NormalTok{,}\DecValTok{1}\SpecialCharTok{:}\DecValTok{3}\NormalTok{), }\FunctionTok{paste0}\NormalTok{(}\StringTok{"ROA"}\NormalTok{,}\DecValTok{1}\SpecialCharTok{:}\DecValTok{3}\NormalTok{), }\FunctionTok{paste0}\NormalTok{(}\StringTok{"CONT"}\NormalTok{,}\DecValTok{1}\SpecialCharTok{:}\DecValTok{3}\NormalTok{))}
\NormalTok{dat\_raw }\OtherTok{\textless{}{-}}\NormalTok{ dat\_raw }\SpecialCharTok{\%\textgreater{}\%} \FunctionTok{mutate}\NormalTok{(}\FunctionTok{across}\NormalTok{(}\FunctionTok{all\_of}\NormalTok{(hpt\_items), }\SpecialCharTok{\textasciitilde{}} \FunctionTok{suppressWarnings}\NormalTok{(}\FunctionTok{as.numeric}\NormalTok{(.))))}

\CommentTok{\# Reverse POP item{-}wise (so higher = more contextualised)}
\NormalTok{POP\_rev\_items }\OtherTok{\textless{}{-}} \FunctionTok{paste0}\NormalTok{(}\StringTok{"POP"}\NormalTok{, }\DecValTok{1}\SpecialCharTok{:}\DecValTok{3}\NormalTok{)}
\NormalTok{dat\_raw }\OtherTok{\textless{}{-}}\NormalTok{ dat\_raw }\SpecialCharTok{\%\textgreater{}\%}
  \FunctionTok{mutate}\NormalTok{(}\FunctionTok{across}\NormalTok{(}\FunctionTok{all\_of}\NormalTok{(POP\_rev\_items), }\SpecialCharTok{\textasciitilde{}} \DecValTok{5} \SpecialCharTok{{-}}\NormalTok{ ., }\AttributeTok{.names =} \StringTok{"\{.col\}\_rev"}\NormalTok{)) }\SpecialCharTok{\%\textgreater{}\%}
  \FunctionTok{mutate}\NormalTok{(}
    \AttributeTok{HPT\_POP\_raw =} \FunctionTok{rowMeans}\NormalTok{(}\FunctionTok{across}\NormalTok{(POP1}\SpecialCharTok{:}\NormalTok{POP3), }\AttributeTok{na.rm =} \ConstantTok{TRUE}\NormalTok{),           }\CommentTok{\# presentism, higher = worse}
    \AttributeTok{HPT\_POP\_rev =} \FunctionTok{rowMeans}\NormalTok{(}\FunctionTok{across}\NormalTok{(}\FunctionTok{paste0}\NormalTok{(POP\_rev\_items, }\StringTok{"\_rev"}\NormalTok{)), }\AttributeTok{na.rm =} \ConstantTok{TRUE}\NormalTok{),  }\CommentTok{\# higher = better}
    \AttributeTok{HPT\_CONT    =} \FunctionTok{rowMeans}\NormalTok{(}\FunctionTok{across}\NormalTok{(CONT1}\SpecialCharTok{:}\NormalTok{CONT3), }\AttributeTok{na.rm =} \ConstantTok{TRUE}\NormalTok{),}
    \AttributeTok{HPT\_ROA     =} \FunctionTok{rowMeans}\NormalTok{(}\FunctionTok{across}\NormalTok{(ROA1}\SpecialCharTok{:}\NormalTok{ROA3),   }\AttributeTok{na.rm =} \ConstantTok{TRUE}\NormalTok{),}

    \CommentTok{\# Canonical composites (CTX6 is our stable default)}
    \AttributeTok{HPT\_CTX6 =} \FunctionTok{rowMeans}\NormalTok{(}\FunctionTok{cbind}\NormalTok{(HPT\_POP\_rev, HPT\_CONT), }\AttributeTok{na.rm =} \ConstantTok{TRUE}\NormalTok{),}
    \AttributeTok{HPT\_TOT9 =} \FunctionTok{rowMeans}\NormalTok{(}\FunctionTok{cbind}\NormalTok{(HPT\_POP\_rev, HPT\_CONT, HPT\_ROA), }\AttributeTok{na.rm =} \ConstantTok{TRUE}\NormalTok{)}
\NormalTok{  )}
\end{Highlighting}
\end{Shaded}

We use variables as defined in the \textbf{codebook} (metadata; KN1-KN6;
POP1-POP3; ROA1-ROA3; CONT1-CONT3; FR-LF mini RD1-RD3 \& NS1-NS3; KSA-3
A1-A3, U1-U3, K1-K3; SDR1-SDR5).

\hypertarget{scoring-grouping}{%
\subsection{Scoring \& grouping}\label{scoring-grouping}}

\begin{itemize}
\tightlist
\item
  \textbf{HPT items} are 1-4. We reverse only \emph{presentism} items
  (\texttt{POP1-POP3}: \texttt{5\ -\ POP*}) so that a single
  higher-is-better direction is used for scale construction and MG-CFA.
  DIF uses original item codings (reversal is not required for DIF
  detection).
\item
  \textbf{Ideology composite}: FR-LF-mini (RD1-3 + NS1-3) and KSA-3 (9
  items). We z-score the two scale means and average them →
  \textbf{IDEO\_Z}. \emph{Low} = bottom 33\%, \emph{High} = top 33\%
  (middle third excluded to sharpen contrasts).
\item
  \textbf{Controls}: prior knowledge (sum KN1-KN6), social desirability
  (SDR1-SDR5; note SDR2--SDR4 are already reversed upstream per
  codebook).
\end{itemize}

\begin{Shaded}
\begin{Highlighting}[]
\CommentTok{\# Select blocks}
\NormalTok{frlf\_items }\OtherTok{\textless{}{-}} \FunctionTok{c}\NormalTok{(}\FunctionTok{paste0}\NormalTok{(}\StringTok{"RD"}\NormalTok{,}\DecValTok{1}\SpecialCharTok{:}\DecValTok{3}\NormalTok{), }\FunctionTok{paste0}\NormalTok{(}\StringTok{"NS"}\NormalTok{,}\DecValTok{1}\SpecialCharTok{:}\DecValTok{3}\NormalTok{))}
\NormalTok{ksa\_items  }\OtherTok{\textless{}{-}} \FunctionTok{c}\NormalTok{(}\FunctionTok{paste0}\NormalTok{(}\StringTok{"A"}\NormalTok{,}\DecValTok{1}\SpecialCharTok{:}\DecValTok{3}\NormalTok{), }\FunctionTok{paste0}\NormalTok{(}\StringTok{"U"}\NormalTok{,}\DecValTok{1}\SpecialCharTok{:}\DecValTok{3}\NormalTok{), }\FunctionTok{paste0}\NormalTok{(}\StringTok{"K"}\NormalTok{,}\DecValTok{1}\SpecialCharTok{:}\DecValTok{3}\NormalTok{))}
\NormalTok{kn\_items   }\OtherTok{\textless{}{-}} \FunctionTok{paste0}\NormalTok{(}\StringTok{"KN"}\NormalTok{,}\DecValTok{1}\SpecialCharTok{:}\DecValTok{6}\NormalTok{)}
\NormalTok{sdr\_items  }\OtherTok{\textless{}{-}} \FunctionTok{paste0}\NormalTok{(}\StringTok{"SDR"}\NormalTok{,}\DecValTok{1}\SpecialCharTok{:}\DecValTok{5}\NormalTok{)}

\CommentTok{\# Coerce predictors to numeric}
\NormalTok{num\_blocks }\OtherTok{\textless{}{-}} \FunctionTok{c}\NormalTok{(frlf\_items, ksa\_items, kn\_items, sdr\_items)}

\NormalTok{dat }\OtherTok{\textless{}{-}}\NormalTok{ dat\_raw }\SpecialCharTok{\%\textgreater{}\%}
  \FunctionTok{mutate}\NormalTok{(}\FunctionTok{across}\NormalTok{(}\FunctionTok{all\_of}\NormalTok{(num\_blocks), }\SpecialCharTok{\textasciitilde{}} \FunctionTok{suppressWarnings}\NormalTok{(}\FunctionTok{as.numeric}\NormalTok{(.)))) }\SpecialCharTok{\%\textgreater{}\%}
  \CommentTok{\# Scale scores}
  \FunctionTok{rowwise}\NormalTok{() }\SpecialCharTok{\%\textgreater{}\%}
  \FunctionTok{mutate}\NormalTok{(}
    \AttributeTok{HPT\_total =} \FunctionTok{mean}\NormalTok{(}\FunctionTok{c\_across}\NormalTok{(}\FunctionTok{c}\NormalTok{(}\FunctionTok{paste0}\NormalTok{(}\StringTok{"POP"}\NormalTok{,}\DecValTok{1}\SpecialCharTok{:}\DecValTok{3}\NormalTok{,}\StringTok{"\_rev"}\NormalTok{), ROA1}\SpecialCharTok{:}\NormalTok{ROA3, CONT1}\SpecialCharTok{:}\NormalTok{CONT3)), }\AttributeTok{na.rm =} \ConstantTok{TRUE}\NormalTok{),}
    \AttributeTok{HPT\_CONT  =} \FunctionTok{mean}\NormalTok{(}\FunctionTok{c\_across}\NormalTok{(CONT1}\SpecialCharTok{:}\NormalTok{CONT3), }\AttributeTok{na.rm =} \ConstantTok{TRUE}\NormalTok{),}
    \AttributeTok{HPT\_ROA   =} \FunctionTok{mean}\NormalTok{(}\FunctionTok{c\_across}\NormalTok{(ROA1}\SpecialCharTok{:}\NormalTok{ROA3), }\AttributeTok{na.rm =} \ConstantTok{TRUE}\NormalTok{),}
    \AttributeTok{HPT\_POPR  =} \FunctionTok{mean}\NormalTok{(}\FunctionTok{c\_across}\NormalTok{(}\FunctionTok{paste0}\NormalTok{(}\StringTok{"POP"}\NormalTok{,}\DecValTok{1}\SpecialCharTok{:}\DecValTok{3}\NormalTok{,}\StringTok{"\_rev"}\NormalTok{)), }\AttributeTok{na.rm =} \ConstantTok{TRUE}\NormalTok{),}
    \AttributeTok{FRLF\_mean =} \FunctionTok{mean}\NormalTok{(}\FunctionTok{c\_across}\NormalTok{(}\FunctionTok{all\_of}\NormalTok{(frlf\_items)), }\AttributeTok{na.rm =} \ConstantTok{TRUE}\NormalTok{),}
    \AttributeTok{KSA\_mean  =} \FunctionTok{mean}\NormalTok{(}\FunctionTok{c\_across}\NormalTok{(}\FunctionTok{all\_of}\NormalTok{(ksa\_items)),  }\AttributeTok{na.rm =} \ConstantTok{TRUE}\NormalTok{),}
    \AttributeTok{KN\_sum    =} \FunctionTok{sum}\NormalTok{(}\FunctionTok{c\_across}\NormalTok{(}\FunctionTok{all\_of}\NormalTok{(kn\_items)),    }\AttributeTok{na.rm =} \ConstantTok{TRUE}\NormalTok{),}
    \AttributeTok{SDR\_mean  =} \FunctionTok{mean}\NormalTok{(}\FunctionTok{c\_across}\NormalTok{(}\FunctionTok{all\_of}\NormalTok{(sdr\_items)),  }\AttributeTok{na.rm =} \ConstantTok{TRUE}\NormalTok{)}
\NormalTok{  ) }\SpecialCharTok{\%\textgreater{}\%}
  \FunctionTok{ungroup}\NormalTok{() }\SpecialCharTok{\%\textgreater{}\%}
  \FunctionTok{mutate}\NormalTok{(}
    \AttributeTok{FRLF\_z =} \FunctionTok{as.numeric}\NormalTok{(}\FunctionTok{scale}\NormalTok{(FRLF\_mean)),}
    \AttributeTok{KSA\_z  =} \FunctionTok{as.numeric}\NormalTok{(}\FunctionTok{scale}\NormalTok{(KSA\_mean)),}
    \AttributeTok{IDEO\_Z =}\NormalTok{ (FRLF\_z }\SpecialCharTok{+}\NormalTok{ KSA\_z) }\SpecialCharTok{/} \DecValTok{2}
\NormalTok{  )}

\CommentTok{\# Define tertile groups}
\NormalTok{qs }\OtherTok{\textless{}{-}} \FunctionTok{quantile}\NormalTok{(dat}\SpecialCharTok{$}\NormalTok{IDEO\_Z, }\AttributeTok{probs =} \FunctionTok{c}\NormalTok{(.}\DecValTok{3334}\NormalTok{, .}\DecValTok{6666}\NormalTok{), }\AttributeTok{na.rm =} \ConstantTok{TRUE}\NormalTok{)}
\NormalTok{dat }\OtherTok{\textless{}{-}}\NormalTok{ dat }\SpecialCharTok{\%\textgreater{}\%}
  \FunctionTok{mutate}\NormalTok{(}
    \AttributeTok{ideology\_group =} \FunctionTok{case\_when}\NormalTok{(}
\NormalTok{      IDEO\_Z }\SpecialCharTok{\textless{}=}\NormalTok{ qs[}\DecValTok{1}\NormalTok{] }\SpecialCharTok{\textasciitilde{}} \StringTok{"Low"}\NormalTok{,}
\NormalTok{      IDEO\_Z }\SpecialCharTok{\textgreater{}=}\NormalTok{ qs[}\DecValTok{2}\NormalTok{] }\SpecialCharTok{\textasciitilde{}} \StringTok{"High"}\NormalTok{,}
      \ConstantTok{TRUE} \SpecialCharTok{\textasciitilde{}} \StringTok{"Mid"}
\NormalTok{    )}
\NormalTok{  )}

\FunctionTok{kable}\NormalTok{(dat }\SpecialCharTok{\%\textgreater{}\%} \FunctionTok{count}\NormalTok{(ideology\_group), }\AttributeTok{caption =} \StringTok{"Group sizes (Low/High ideology; Mid excluded from group{-}wise tests)"}\NormalTok{)}
\end{Highlighting}
\end{Shaded}

\begin{longtable}[]{@{}lr@{}}
\caption{Group sizes (Low/High ideology; Mid excluded from group-wise
tests)}\tabularnewline
\toprule\noalign{}
ideology\_group & n \\
\midrule\noalign{}
\endfirsthead
\toprule\noalign{}
ideology\_group & n \\
\midrule\noalign{}
\endhead
\bottomrule\noalign{}
\endlastfoot
High & 77 \\
Low & 77 \\
Mid & 80 \\
\end{longtable}

\begin{quote}
\textbf{Note.} We focus on \emph{Low} vs \emph{High} to maximise
contrast for DIF/MG-CFA. \emph{Mid} is retained for descriptives only.
\end{quote}

\hypertarget{descriptives-checks}{%
\section{Descriptives (checks)}\label{descriptives-checks}}

\begin{Shaded}
\begin{Highlighting}[]
\NormalTok{desc\_tbl }\OtherTok{\textless{}{-}}\NormalTok{ dat }\SpecialCharTok{\%\textgreater{}\%}
  \FunctionTok{group\_by}\NormalTok{(ideology\_group) }\SpecialCharTok{\%\textgreater{}\%}
  \FunctionTok{summarise}\NormalTok{(}\AttributeTok{n =} \FunctionTok{n}\NormalTok{(),}
            \AttributeTok{HPT\_total =} \FunctionTok{mean}\NormalTok{(HPT\_total, }\AttributeTok{na.rm =} \ConstantTok{TRUE}\NormalTok{),}
            \AttributeTok{KN\_sum    =} \FunctionTok{mean}\NormalTok{(KN\_sum,    }\AttributeTok{na.rm =} \ConstantTok{TRUE}\NormalTok{),}
            \AttributeTok{SDR\_mean  =} \FunctionTok{mean}\NormalTok{(SDR\_mean,  }\AttributeTok{na.rm =} \ConstantTok{TRUE}\NormalTok{)) }\SpecialCharTok{\%\textgreater{}\%}
  \FunctionTok{arrange}\NormalTok{(}\FunctionTok{match}\NormalTok{(ideology\_group, }\FunctionTok{c}\NormalTok{(}\StringTok{"Low"}\NormalTok{,}\StringTok{"Mid"}\NormalTok{,}\StringTok{"High"}\NormalTok{)))}
\FunctionTok{kable}\NormalTok{(desc\_tbl, }\AttributeTok{digits =} \DecValTok{2}\NormalTok{, }\AttributeTok{caption =} \StringTok{"Descriptives by ideology group (means)"}\NormalTok{)}
\end{Highlighting}
\end{Shaded}

\begin{longtable}[]{@{}lrrrr@{}}
\caption{Descriptives by ideology group (means)}\tabularnewline
\toprule\noalign{}
ideology\_group & n & HPT\_total & KN\_sum & SDR\_mean \\
\midrule\noalign{}
\endfirsthead
\toprule\noalign{}
ideology\_group & n & HPT\_total & KN\_sum & SDR\_mean \\
\midrule\noalign{}
\endhead
\bottomrule\noalign{}
\endlastfoot
Low & 77 & 2.92 & 3.47 & 3.16 \\
Mid & 80 & 2.86 & 2.98 & 3.03 \\
High & 77 & 2.75 & 3.03 & 2.83 \\
\end{longtable}

\hypertarget{dif-analysis-ordinal-item-by-item}{%
\section{DIF analysis (ordinal,
item-by-item)}\label{dif-analysis-ordinal-item-by-item}}

\textbf{Goal.} At \emph{equal HPT ability}, do Low/High ideology groups
respond differently to specific items? We match on total HPT (item-rest)
and test both uniform and non-uniform DIF per item (α = .01, Bonferroni
adjusted).

\begin{Shaded}
\begin{Highlighting}[]
\DocumentationTok{\#\# Keep only Low/High groups}
\NormalTok{anal }\OtherTok{\textless{}{-}}\NormalTok{ dat[dat}\SpecialCharTok{$}\NormalTok{ideology\_group }\SpecialCharTok{\%in\%} \FunctionTok{c}\NormalTok{(}\StringTok{"Low"}\NormalTok{,}\StringTok{"High"}\NormalTok{), , drop }\OtherTok{=} \ConstantTok{FALSE}\NormalTok{]}

\DocumentationTok{\#\# HPT item list in original coding}
\NormalTok{hpt\_items }\OtherTok{\textless{}{-}} \FunctionTok{c}\NormalTok{(}\StringTok{"POP1"}\NormalTok{,}\StringTok{"POP2"}\NormalTok{,}\StringTok{"POP3"}\NormalTok{,}\StringTok{"ROA1"}\NormalTok{,}\StringTok{"ROA2"}\NormalTok{,}\StringTok{"ROA3"}\NormalTok{,}\StringTok{"CONT1"}\NormalTok{,}\StringTok{"CONT2"}\NormalTok{,}\StringTok{"CONT3"}\NormalTok{)}
\FunctionTok{stopifnot}\NormalTok{(}\FunctionTok{all}\NormalTok{(hpt\_items }\SpecialCharTok{\%in\%} \FunctionTok{names}\NormalTok{(anal)))}

\DocumentationTok{\#\# Build item matrix}
\NormalTok{hpt\_mat }\OtherTok{\textless{}{-}}\NormalTok{ anal[, hpt\_items, drop }\OtherTok{=} \ConstantTok{FALSE}\NormalTok{]}
\ControlFlowTok{for}\NormalTok{ (j }\ControlFlowTok{in} \FunctionTok{seq\_along}\NormalTok{(hpt\_items)) hpt\_mat[[j]] }\OtherTok{\textless{}{-}} \FunctionTok{suppressWarnings}\NormalTok{(}\FunctionTok{as.numeric}\NormalTok{(hpt\_mat[[j]]))}

\DocumentationTok{\#\# Group factor}
\NormalTok{grp }\OtherTok{\textless{}{-}} \FunctionTok{factor}\NormalTok{(anal}\SpecialCharTok{$}\NormalTok{ideology\_group, }\AttributeTok{levels =} \FunctionTok{c}\NormalTok{(}\StringTok{"Low"}\NormalTok{,}\StringTok{"High"}\NormalTok{))}

\DocumentationTok{\#\# Drop rows with \textless{}2 answered items}
\NormalTok{keep }\OtherTok{\textless{}{-}} \FunctionTok{rowSums}\NormalTok{(}\SpecialCharTok{!}\FunctionTok{is.na}\NormalTok{(hpt\_mat)) }\SpecialCharTok{\textgreater{}=} \DecValTok{2}
\NormalTok{hpt\_mat }\OtherTok{\textless{}{-}}\NormalTok{ hpt\_mat[keep, , drop }\OtherTok{=} \ConstantTok{FALSE}\NormalTok{]}
\NormalTok{grp     }\OtherTok{\textless{}{-}} \FunctionTok{droplevels}\NormalTok{(grp[keep])}

\FunctionTok{stopifnot}\NormalTok{(}\FunctionTok{nrow}\NormalTok{(hpt\_mat) }\SpecialCharTok{==} \FunctionTok{length}\NormalTok{(grp), }\FunctionTok{nlevels}\NormalTok{(grp) }\SpecialCharTok{==} \DecValTok{2}\NormalTok{)}
\FunctionTok{print}\NormalTok{(}\FunctionTok{table}\NormalTok{(grp))}
\end{Highlighting}
\end{Shaded}

\begin{verbatim}
## grp
##  Low High 
##   76   77
\end{verbatim}

\begin{Shaded}
\begin{Highlighting}[]
\CommentTok{\# Constrained multi{-}group graded model, then DIF with scheme="drop"}
\NormalTok{mod\_base }\OtherTok{\textless{}{-}} \FunctionTok{multipleGroup}\NormalTok{(}
  \AttributeTok{data       =}\NormalTok{ hpt\_mat,}
  \AttributeTok{model      =} \DecValTok{1}\NormalTok{,}
  \AttributeTok{group      =}\NormalTok{ grp,}
  \AttributeTok{itemtype   =} \StringTok{"graded"}\NormalTok{,}
  \AttributeTok{invariance =} \FunctionTok{c}\NormalTok{(}\StringTok{"slopes"}\NormalTok{, }\StringTok{"intercepts"}\NormalTok{, }\StringTok{"free\_means"}\NormalTok{, }\StringTok{"free\_var"}\NormalTok{)}
\NormalTok{)}
\end{Highlighting}
\end{Shaded}

\begin{Shaded}
\begin{Highlighting}[]
\NormalTok{params\_all }\OtherTok{\textless{}{-}}\NormalTok{ mirt}\SpecialCharTok{::}\FunctionTok{mod2values}\NormalTok{(mod\_base)}\SpecialCharTok{$}\NormalTok{name}
\NormalTok{unique\_pars }\OtherTok{\textless{}{-}} \FunctionTok{unique}\NormalTok{(params\_all)}
\NormalTok{pars\_slope }\OtherTok{\textless{}{-}} \FunctionTok{grep}\NormalTok{(}\StringTok{"\^{}a"}\NormalTok{, unique\_pars, }\AttributeTok{value =} \ConstantTok{TRUE}\NormalTok{)}
\NormalTok{pars\_thr   }\OtherTok{\textless{}{-}} \FunctionTok{grep}\NormalTok{(}\StringTok{"\^{}d}\SpecialCharTok{\textbackslash{}\textbackslash{}}\StringTok{d+$"}\NormalTok{, unique\_pars, }\AttributeTok{value =} \ConstantTok{TRUE}\NormalTok{)}
\FunctionTok{stopifnot}\NormalTok{(}\FunctionTok{length}\NormalTok{(pars\_slope) }\SpecialCharTok{\textgreater{}} \DecValTok{0}\NormalTok{, }\FunctionTok{length}\NormalTok{(pars\_thr) }\SpecialCharTok{\textgreater{}} \DecValTok{0}\NormalTok{)}

\NormalTok{pars\_to\_test }\OtherTok{\textless{}{-}} \FunctionTok{c}\NormalTok{(pars\_slope, pars\_thr)}

\NormalTok{dif\_out }\OtherTok{\textless{}{-}} \FunctionTok{DIF}\NormalTok{(}
\NormalTok{  mod\_base,}
  \AttributeTok{which.par   =}\NormalTok{ pars\_to\_test,}
  \AttributeTok{scheme      =} \StringTok{"drop"}\NormalTok{,}
  \AttributeTok{items2test  =} \FunctionTok{colnames}\NormalTok{(hpt\_mat),}
  \AttributeTok{p.adjust    =} \StringTok{"bonferroni"}\NormalTok{,}
  \AttributeTok{verbose     =} \ConstantTok{FALSE}
\NormalTok{)}

\NormalTok{res\_tbl }\OtherTok{\textless{}{-}} \FunctionTok{as.data.frame}\NormalTok{(dif\_out)}
\NormalTok{res\_tbl}\SpecialCharTok{$}\NormalTok{Item }\OtherTok{\textless{}{-}} \FunctionTok{rownames}\NormalTok{(res\_tbl)}

\NormalTok{get\_min\_p }\OtherTok{\textless{}{-}} \ControlFlowTok{function}\NormalTok{(df, prefix\_list) \{}
\NormalTok{  cols }\OtherTok{\textless{}{-}} \FunctionTok{unlist}\NormalTok{(}\FunctionTok{lapply}\NormalTok{(prefix\_list, }\ControlFlowTok{function}\NormalTok{(p) }\FunctionTok{grep}\NormalTok{(}\FunctionTok{paste0}\NormalTok{(}\StringTok{"\^{}p}\SpecialCharTok{\textbackslash{}\textbackslash{}}\StringTok{."}\NormalTok{, p, }\StringTok{"$"}\NormalTok{), }\FunctionTok{names}\NormalTok{(df), }\AttributeTok{value=}\ConstantTok{TRUE}\NormalTok{)))}
  \ControlFlowTok{if}\NormalTok{ (}\FunctionTok{length}\NormalTok{(cols) }\SpecialCharTok{==} \DecValTok{0}\NormalTok{) }\FunctionTok{return}\NormalTok{(}\FunctionTok{rep}\NormalTok{(}\ConstantTok{NA}\NormalTok{, }\FunctionTok{nrow}\NormalTok{(df)))}
  \FunctionTok{apply}\NormalTok{(df[, cols, }\AttributeTok{drop=}\ConstantTok{FALSE}\NormalTok{], }\DecValTok{1}\NormalTok{, min, }\AttributeTok{na.rm =} \ConstantTok{TRUE}\NormalTok{)}
\NormalTok{\}}

\NormalTok{alpha }\OtherTok{\textless{}{-}} \FloatTok{0.01}

\NormalTok{res\_tbl }\OtherTok{\textless{}{-}}\NormalTok{ res\_tbl }\SpecialCharTok{\%\textgreater{}\%}
  \FunctionTok{mutate}\NormalTok{(}
    \AttributeTok{p\_nonuniform =} \FunctionTok{get\_min\_p}\NormalTok{(}\FunctionTok{cur\_data}\NormalTok{(), pars\_slope),}
    \AttributeTok{p\_uniform    =} \FunctionTok{get\_min\_p}\NormalTok{(}\FunctionTok{cur\_data}\NormalTok{(), pars\_thr),}
    \AttributeTok{Flag\_nonuniform =} \FunctionTok{ifelse}\NormalTok{(}\SpecialCharTok{!}\FunctionTok{is.na}\NormalTok{(p\_nonuniform) }\SpecialCharTok{\&}\NormalTok{ p\_nonuniform }\SpecialCharTok{\textless{}}\NormalTok{ alpha, }\StringTok{"YES"}\NormalTok{, }\StringTok{"no"}\NormalTok{),}
    \AttributeTok{Flag\_uniform    =} \FunctionTok{ifelse}\NormalTok{(}\SpecialCharTok{!}\FunctionTok{is.na}\NormalTok{(p\_uniform)    }\SpecialCharTok{\&}\NormalTok{ p\_uniform    }\SpecialCharTok{\textless{}}\NormalTok{ alpha, }\StringTok{"YES"}\NormalTok{, }\StringTok{"no"}\NormalTok{)}
\NormalTok{  ) }\SpecialCharTok{\%\textgreater{}\%}
  \FunctionTok{select}\NormalTok{(Item, p\_nonuniform, p\_uniform, Flag\_nonuniform, Flag\_uniform)}

\FunctionTok{kable}\NormalTok{(res\_tbl, }\AttributeTok{digits =} \DecValTok{4}\NormalTok{,}
      \AttributeTok{caption =} \StringTok{"DIF per item (mirt; graded). Non{-}uniform = Slope (a1); Uniform = Any Threshold (d1{-}dK). Bonferroni alpha = 0.01."}\NormalTok{)}
\end{Highlighting}
\end{Shaded}

\begin{longtable}[]{@{}llllll@{}}
\caption{DIF per item (mirt; graded). Non-uniform = Slope (a1); Uniform
= Any Threshold (d1-dK). Bonferroni alpha = 0.01.}\tabularnewline
\toprule\noalign{}
& Item & p\_nonuniform & p\_uniform & Flag\_nonuniform &
Flag\_uniform \\
\midrule\noalign{}
\endfirsthead
\toprule\noalign{}
& Item & p\_nonuniform & p\_uniform & Flag\_nonuniform &
Flag\_uniform \\
\midrule\noalign{}
\endhead
\bottomrule\noalign{}
\endlastfoot
POP1 & POP1 & NA & NA & no & no \\
POP2 & POP2 & NA & NA & no & no \\
POP3 & POP3 & NA & NA & no & no \\
ROA1 & ROA1 & NA & NA & no & no \\
ROA2 & ROA2 & NA & NA & no & no \\
ROA3 & ROA3 & NA & NA & no & no \\
CONT1 & CONT1 & NA & NA & no & no \\
CONT2 & CONT2 & NA & NA & no & no \\
CONT3 & CONT3 & NA & NA & no & no \\
\end{longtable}

\begin{Shaded}
\begin{Highlighting}[]
\NormalTok{flagged }\OtherTok{\textless{}{-}} \FunctionTok{with}\NormalTok{(res\_tbl, Item[Flag\_uniform }\SpecialCharTok{==} \StringTok{"YES"} \SpecialCharTok{|}\NormalTok{ Flag\_nonuniform }\SpecialCharTok{==} \StringTok{"YES"}\NormalTok{])}
\ControlFlowTok{if}\NormalTok{ (}\FunctionTok{length}\NormalTok{(flagged) }\SpecialCharTok{\textgreater{}} \DecValTok{0}\NormalTok{) \{}
\NormalTok{  which\_item }\OtherTok{\textless{}{-}} \FunctionTok{which}\NormalTok{(}\FunctionTok{colnames}\NormalTok{(hpt\_mat) }\SpecialCharTok{==}\NormalTok{ flagged[}\DecValTok{1}\NormalTok{])}
  \FunctionTok{plot}\NormalTok{(mod\_base, }\AttributeTok{type =} \StringTok{"trace"}\NormalTok{, }\AttributeTok{which.items =}\NormalTok{ which\_item,}
       \AttributeTok{facet\_items =} \ConstantTok{FALSE}\NormalTok{, }\AttributeTok{groups =} \FunctionTok{levels}\NormalTok{(grp))}
\NormalTok{\} }\ControlFlowTok{else}\NormalTok{ \{}
  \FunctionTok{plot.new}\NormalTok{(); }\FunctionTok{text}\NormalTok{(}\FloatTok{0.5}\NormalTok{, }\FloatTok{0.5}\NormalTok{, }\StringTok{"No DIF{-}flagged item at alpha = 0.01."}\NormalTok{)}
\NormalTok{\}}
\end{Highlighting}
\end{Shaded}

\includegraphics{/home/yetty/Projects/phd-029-hpt-and-extremism/outputs/04_dif-and-mg-cfa-hpt-bias_files/figure-latex/DIF-plot-1.pdf}

\hypertarget{multi-group-cfa-configural-metric-scalar}{%
\section{Multi-group CFA (configural → metric →
scalar)}\label{multi-group-cfa-configural-metric-scalar}}

\textbf{Model.} We specify a \textbf{three-factor model} (POP\_rev, ROA,
CONT) with POP items reversed so that all loadings point to \emph{more
HPT-congruent} responses. We then test invariance across Low vs High
ideology groups.

\begin{Shaded}
\begin{Highlighting}[]
\CommentTok{\# Build analysis frame with reversed POP and intact ROA/CONT}
\NormalTok{cfad }\OtherTok{\textless{}{-}}\NormalTok{ dat }\SpecialCharTok{\%\textgreater{}\%}
  \FunctionTok{filter}\NormalTok{(ideology\_group }\SpecialCharTok{\%in\%} \FunctionTok{c}\NormalTok{(}\StringTok{"Low"}\NormalTok{,}\StringTok{"High"}\NormalTok{)) }\SpecialCharTok{\%\textgreater{}\%}
  \FunctionTok{transmute}\NormalTok{(}
\NormalTok{    ideology\_group, class\_id,         }\CommentTok{\# keep cluster id for reference (not used by lavaan here)}
    \AttributeTok{POP1 =} \DecValTok{5} \SpecialCharTok{{-}}\NormalTok{ POP1,}
    \AttributeTok{POP2 =} \DecValTok{5} \SpecialCharTok{{-}}\NormalTok{ POP2,}
    \AttributeTok{POP3 =} \DecValTok{5} \SpecialCharTok{{-}}\NormalTok{ POP3,}
\NormalTok{    ROA1, ROA2, ROA3,}
\NormalTok{    CONT1, CONT2, CONT3}
\NormalTok{  )}

\CommentTok{\# Coerce all item columns to numeric and enforce ordinal 1:4 range; replace out{-}of{-}range with NA}
\NormalTok{ord\_items }\OtherTok{\textless{}{-}} \FunctionTok{setdiff}\NormalTok{(}\FunctionTok{names}\NormalTok{(cfad), }\FunctionTok{c}\NormalTok{(}\StringTok{"ideology\_group"}\NormalTok{, }\StringTok{"class\_id"}\NormalTok{))}
\NormalTok{cfad }\OtherTok{\textless{}{-}}\NormalTok{ cfad }\SpecialCharTok{\%\textgreater{}\%} \FunctionTok{mutate}\NormalTok{(}\FunctionTok{across}\NormalTok{(}\FunctionTok{all\_of}\NormalTok{(ord\_items), }\SpecialCharTok{\textasciitilde{}} \FunctionTok{suppressWarnings}\NormalTok{(}\FunctionTok{as.numeric}\NormalTok{(.))))}
\NormalTok{cfad }\OtherTok{\textless{}{-}}\NormalTok{ cfad }\SpecialCharTok{\%\textgreater{}\%} \FunctionTok{mutate}\NormalTok{(}\FunctionTok{across}\NormalTok{(}\FunctionTok{all\_of}\NormalTok{(ord\_items), }\SpecialCharTok{\textasciitilde{}} \FunctionTok{ifelse}\NormalTok{(. }\SpecialCharTok{\%in\%} \DecValTok{1}\SpecialCharTok{:}\DecValTok{4}\NormalTok{, ., }\ConstantTok{NA\_real\_}\NormalTok{)))}

\CommentTok{\# If any columns had non 1{-}4 values, warn instead of stopping}
\NormalTok{bad\_cols }\OtherTok{\textless{}{-}} \FunctionTok{vapply}\NormalTok{(cfad[ord\_items], }\ControlFlowTok{function}\NormalTok{(x) }\FunctionTok{any}\NormalTok{(}\SpecialCharTok{!}\FunctionTok{is.na}\NormalTok{(x) }\SpecialCharTok{\&} \SpecialCharTok{!}\NormalTok{(x }\SpecialCharTok{\%in\%} \DecValTok{1}\SpecialCharTok{:}\DecValTok{4}\NormalTok{)), }\FunctionTok{logical}\NormalTok{(}\DecValTok{1}\NormalTok{))}
\ControlFlowTok{if}\NormalTok{ (}\FunctionTok{any}\NormalTok{(bad\_cols)) \{}
  \FunctionTok{warning}\NormalTok{(}\FunctionTok{sprintf}\NormalTok{(}\StringTok{"Non{-}1:4 values were set to NA in: \%s"}\NormalTok{, }\FunctionTok{paste}\NormalTok{(}\FunctionTok{names}\NormalTok{(bad\_cols)[bad\_cols], }\AttributeTok{collapse=}\StringTok{", "}\NormalTok{)))}
\NormalTok{\}}
\end{Highlighting}
\end{Shaded}

\begin{Shaded}
\begin{Highlighting}[]
\NormalTok{model\_3f }\OtherTok{\textless{}{-}} \StringTok{\textquotesingle{}}
\StringTok{  POP =\textasciitilde{} POP1 + POP2 + POP3}
\StringTok{  ROA =\textasciitilde{} ROA1 + ROA2 + ROA3}
\StringTok{  CONT =\textasciitilde{} CONT1 + CONT2 + CONT3}
\StringTok{\textquotesingle{}}
\end{Highlighting}
\end{Shaded}

\begin{Shaded}
\begin{Highlighting}[]
\CommentTok{\# Run invariance ladder with WLSMV on ordered items; DO NOT pass cluster (not supported with ordered)}
\NormalTok{fit\_conf }\OtherTok{\textless{}{-}} \FunctionTok{cfa}\NormalTok{(model\_3f, }\AttributeTok{data =}\NormalTok{ cfad, }\AttributeTok{group =} \StringTok{"ideology\_group"}\NormalTok{,}
                \AttributeTok{ordered =}\NormalTok{ ord\_items, }\AttributeTok{estimator =} \StringTok{"WLSMV"}\NormalTok{)}

\NormalTok{fit\_metr }\OtherTok{\textless{}{-}} \FunctionTok{cfa}\NormalTok{(model\_3f, }\AttributeTok{data =}\NormalTok{ cfad, }\AttributeTok{group =} \StringTok{"ideology\_group"}\NormalTok{,}
                \AttributeTok{ordered =}\NormalTok{ ord\_items, }\AttributeTok{estimator =} \StringTok{"WLSMV"}\NormalTok{,}
                \AttributeTok{group.equal =} \FunctionTok{c}\NormalTok{(}\StringTok{"loadings"}\NormalTok{))}

\NormalTok{fit\_scal }\OtherTok{\textless{}{-}} \FunctionTok{cfa}\NormalTok{(model\_3f, }\AttributeTok{data =}\NormalTok{ cfad, }\AttributeTok{group =} \StringTok{"ideology\_group"}\NormalTok{,}
                \AttributeTok{ordered =}\NormalTok{ ord\_items, }\AttributeTok{estimator =} \StringTok{"WLSMV"}\NormalTok{,}
                \AttributeTok{group.equal =} \FunctionTok{c}\NormalTok{(}\StringTok{"loadings"}\NormalTok{, }\StringTok{"thresholds"}\NormalTok{))}

\NormalTok{get\_fit }\OtherTok{\textless{}{-}} \ControlFlowTok{function}\NormalTok{(fit) \{}
  \FunctionTok{fitMeasures}\NormalTok{(fit, }\FunctionTok{c}\NormalTok{(}\StringTok{"chisq.scaled"}\NormalTok{, }\StringTok{"df.scaled"}\NormalTok{, }\StringTok{"pvalue.scaled"}\NormalTok{,}
                    \StringTok{"cfi.scaled"}\NormalTok{, }\StringTok{"rmsea.scaled"}\NormalTok{, }\StringTok{"srmr"}\NormalTok{))}
\NormalTok{\}}

\NormalTok{fits }\OtherTok{\textless{}{-}} \FunctionTok{bind\_rows}\NormalTok{(}
  \AttributeTok{Configural =} \FunctionTok{get\_fit}\NormalTok{(fit\_conf),}
  \AttributeTok{Metric     =} \FunctionTok{get\_fit}\NormalTok{(fit\_metr),}
  \AttributeTok{Scalar     =} \FunctionTok{get\_fit}\NormalTok{(fit\_scal)}
\NormalTok{) }\SpecialCharTok{\%\textgreater{}\%}
  \FunctionTok{mutate}\NormalTok{(}\AttributeTok{Model =} \FunctionTok{c}\NormalTok{(}\StringTok{"Configural"}\NormalTok{, }\StringTok{"Metric"}\NormalTok{, }\StringTok{"Scalar"}\NormalTok{)) }\SpecialCharTok{\%\textgreater{}\%}
  \FunctionTok{select}\NormalTok{(Model, }\FunctionTok{everything}\NormalTok{())}

\NormalTok{fits }\SpecialCharTok{\%\textgreater{}\%} \FunctionTok{mutate}\NormalTok{(}\FunctionTok{across}\NormalTok{(}\FunctionTok{where}\NormalTok{(is.numeric), round, }\DecValTok{3}\NormalTok{)) }\SpecialCharTok{\%\textgreater{}\%}
  \FunctionTok{kable}\NormalTok{(}\AttributeTok{caption =} \StringTok{"MG{-}CFA fit indices by invariance level (WLSMV)."}\NormalTok{)}
\end{Highlighting}
\end{Shaded}

\begin{longtable}[]{@{}
  >{\raggedright\arraybackslash}p{(\columnwidth - 12\tabcolsep) * \real{0.1410}}
  >{\raggedleft\arraybackslash}p{(\columnwidth - 12\tabcolsep) * \real{0.1667}}
  >{\raggedleft\arraybackslash}p{(\columnwidth - 12\tabcolsep) * \real{0.1282}}
  >{\raggedleft\arraybackslash}p{(\columnwidth - 12\tabcolsep) * \real{0.1795}}
  >{\raggedleft\arraybackslash}p{(\columnwidth - 12\tabcolsep) * \real{0.1410}}
  >{\raggedleft\arraybackslash}p{(\columnwidth - 12\tabcolsep) * \real{0.1667}}
  >{\raggedleft\arraybackslash}p{(\columnwidth - 12\tabcolsep) * \real{0.0769}}@{}}
\caption{MG-CFA fit indices by invariance level (WLSMV).}\tabularnewline
\toprule\noalign{}
\begin{minipage}[b]{\linewidth}\raggedright
Model
\end{minipage} & \begin{minipage}[b]{\linewidth}\raggedleft
chisq.scaled
\end{minipage} & \begin{minipage}[b]{\linewidth}\raggedleft
df.scaled
\end{minipage} & \begin{minipage}[b]{\linewidth}\raggedleft
pvalue.scaled
\end{minipage} & \begin{minipage}[b]{\linewidth}\raggedleft
cfi.scaled
\end{minipage} & \begin{minipage}[b]{\linewidth}\raggedleft
rmsea.scaled
\end{minipage} & \begin{minipage}[b]{\linewidth}\raggedleft
srmr
\end{minipage} \\
\midrule\noalign{}
\endfirsthead
\toprule\noalign{}
\begin{minipage}[b]{\linewidth}\raggedright
Model
\end{minipage} & \begin{minipage}[b]{\linewidth}\raggedleft
chisq.scaled
\end{minipage} & \begin{minipage}[b]{\linewidth}\raggedleft
df.scaled
\end{minipage} & \begin{minipage}[b]{\linewidth}\raggedleft
pvalue.scaled
\end{minipage} & \begin{minipage}[b]{\linewidth}\raggedleft
cfi.scaled
\end{minipage} & \begin{minipage}[b]{\linewidth}\raggedleft
rmsea.scaled
\end{minipage} & \begin{minipage}[b]{\linewidth}\raggedleft
srmr
\end{minipage} \\
\midrule\noalign{}
\endhead
\bottomrule\noalign{}
\endlastfoot
Configural & 64.608 & 48 & 0.055 & 0.950 & 0.070 & 0.089 \\
Metric & 79.964 & 54 & 0.012 & 0.921 & 0.082 & 0.103 \\
Scalar & 78.986 & 63 & 0.084 & 0.952 & 0.060 & 0.092 \\
\end{longtable}

\begin{Shaded}
\begin{Highlighting}[]
\NormalTok{deltas }\OtherTok{\textless{}{-}} \FunctionTok{tibble}\NormalTok{(}
  \AttributeTok{step  =} \FunctionTok{c}\NormalTok{(}\StringTok{"Configural {-}\textgreater{} Metric"}\NormalTok{, }\StringTok{"Metric {-}\textgreater{} Scalar"}\NormalTok{),}
  \AttributeTok{dCFI  =} \FunctionTok{c}\NormalTok{(fits}\SpecialCharTok{$}\NormalTok{cfi.scaled[}\DecValTok{2}\NormalTok{] }\SpecialCharTok{{-}}\NormalTok{ fits}\SpecialCharTok{$}\NormalTok{cfi.scaled[}\DecValTok{1}\NormalTok{], fits}\SpecialCharTok{$}\NormalTok{cfi.scaled[}\DecValTok{3}\NormalTok{] }\SpecialCharTok{{-}}\NormalTok{ fits}\SpecialCharTok{$}\NormalTok{cfi.scaled[}\DecValTok{2}\NormalTok{]),}
  \AttributeTok{dRMSEA=} \FunctionTok{c}\NormalTok{(fits}\SpecialCharTok{$}\NormalTok{rmsea.scaled[}\DecValTok{2}\NormalTok{] }\SpecialCharTok{{-}}\NormalTok{ fits}\SpecialCharTok{$}\NormalTok{rmsea.scaled[}\DecValTok{1}\NormalTok{], fits}\SpecialCharTok{$}\NormalTok{rmsea.scaled[}\DecValTok{3}\NormalTok{] }\SpecialCharTok{{-}}\NormalTok{ fits}\SpecialCharTok{$}\NormalTok{rmsea.scaled[}\DecValTok{2}\NormalTok{])}
\NormalTok{)}

\NormalTok{deltas }\SpecialCharTok{\%\textgreater{}\%} \FunctionTok{mutate}\NormalTok{(}\FunctionTok{across}\NormalTok{(}\FunctionTok{where}\NormalTok{(is.numeric), round, }\DecValTok{3}\NormalTok{)) }\SpecialCharTok{\%\textgreater{}\%}
  \FunctionTok{kable}\NormalTok{(}\AttributeTok{caption =} \StringTok{"Delta fit (CFI, RMSEA) across steps."}\NormalTok{)}
\end{Highlighting}
\end{Shaded}

\begin{longtable}[]{@{}lrr@{}}
\caption{Delta fit (CFI, RMSEA) across steps.}\tabularnewline
\toprule\noalign{}
step & dCFI & dRMSEA \\
\midrule\noalign{}
\endfirsthead
\toprule\noalign{}
step & dCFI & dRMSEA \\
\midrule\noalign{}
\endhead
\bottomrule\noalign{}
\endlastfoot
Configural -\textgreater{} Metric & -0.028 & 0.012 \\
Metric -\textgreater{} Scalar & 0.030 & -0.022 \\
\end{longtable}

\begin{quote}
\textbf{How to read this.} If \textbf{metric holds} (small ΔCFI/ΔRMSEA),
loadings are equivalent. If \textbf{scalar fails}, thresholds differ →
\textbf{biased group mean comparisons}, supporting H1.
\end{quote}

\hypertarget{optional-two-factor-robustness-check}{%
\section{(Optional) Two-factor robustness
check}\label{optional-two-factor-robustness-check}}

\begin{Shaded}
\begin{Highlighting}[]
\NormalTok{model\_2f }\OtherTok{\textless{}{-}} \StringTok{\textquotesingle{}}
\StringTok{  CTX =\textasciitilde{} POP1 + POP2 + POP3 + CONT1 + CONT2 + CONT3}
\StringTok{  ROA =\textasciitilde{} ROA1 + ROA2 + ROA3}
\StringTok{\textquotesingle{}}
\FunctionTok{measurementInvariance}\NormalTok{(model\_2f, }\AttributeTok{data =}\NormalTok{ cfad, }\AttributeTok{group =} \StringTok{"ideology\_group"}\NormalTok{,}
                      \AttributeTok{estimator =} \StringTok{"WLSMV"}\NormalTok{,}
                      \AttributeTok{ordered =}\NormalTok{ ord\_items)}
\end{Highlighting}
\end{Shaded}

\hypertarget{interpretation-reporting}{%
\section{Interpretation \& reporting}\label{interpretation-reporting}}

\hypertarget{dif-summary}{%
\subsection{DIF summary}\label{dif-summary}}

\begin{Shaded}
\begin{Highlighting}[]
\FunctionTok{kable}\NormalTok{(res\_tbl, }\AttributeTok{caption =} \StringTok{"DIF results (for reference in text)."}\NormalTok{)}
\end{Highlighting}
\end{Shaded}

\begin{longtable}[]{@{}llllll@{}}
\caption{DIF results (for reference in text).}\tabularnewline
\toprule\noalign{}
& Item & p\_nonuniform & p\_uniform & Flag\_nonuniform &
Flag\_uniform \\
\midrule\noalign{}
\endfirsthead
\toprule\noalign{}
& Item & p\_nonuniform & p\_uniform & Flag\_nonuniform &
Flag\_uniform \\
\midrule\noalign{}
\endhead
\bottomrule\noalign{}
\endlastfoot
POP1 & POP1 & NA & NA & no & no \\
POP2 & POP2 & NA & NA & no & no \\
POP3 & POP3 & NA & NA & no & no \\
ROA1 & ROA1 & NA & NA & no & no \\
ROA2 & ROA2 & NA & NA & no & no \\
ROA3 & ROA3 & NA & NA & no & no \\
CONT1 & CONT1 & NA & NA & no & no \\
CONT2 & CONT2 & NA & NA & no & no \\
CONT3 & CONT3 & NA & NA & no & no \\
\end{longtable}

\hypertarget{mg-cfa-summary}{%
\subsection{MG-CFA summary}\label{mg-cfa-summary}}

\begin{Shaded}
\begin{Highlighting}[]
\FunctionTok{kable}\NormalTok{(fits }\SpecialCharTok{\%\textgreater{}\%} \FunctionTok{mutate}\NormalTok{(}\FunctionTok{across}\NormalTok{(}\FunctionTok{where}\NormalTok{(is.numeric), round, }\DecValTok{3}\NormalTok{)),}
      \AttributeTok{caption =} \StringTok{"MG{-}CFA fit to reference in text."}\NormalTok{)}
\end{Highlighting}
\end{Shaded}

\begin{longtable}[]{@{}
  >{\raggedright\arraybackslash}p{(\columnwidth - 12\tabcolsep) * \real{0.1410}}
  >{\raggedleft\arraybackslash}p{(\columnwidth - 12\tabcolsep) * \real{0.1667}}
  >{\raggedleft\arraybackslash}p{(\columnwidth - 12\tabcolsep) * \real{0.1282}}
  >{\raggedleft\arraybackslash}p{(\columnwidth - 12\tabcolsep) * \real{0.1795}}
  >{\raggedleft\arraybackslash}p{(\columnwidth - 12\tabcolsep) * \real{0.1410}}
  >{\raggedleft\arraybackslash}p{(\columnwidth - 12\tabcolsep) * \real{0.1667}}
  >{\raggedleft\arraybackslash}p{(\columnwidth - 12\tabcolsep) * \real{0.0769}}@{}}
\caption{MG-CFA fit to reference in text.}\tabularnewline
\toprule\noalign{}
\begin{minipage}[b]{\linewidth}\raggedright
Model
\end{minipage} & \begin{minipage}[b]{\linewidth}\raggedleft
chisq.scaled
\end{minipage} & \begin{minipage}[b]{\linewidth}\raggedleft
df.scaled
\end{minipage} & \begin{minipage}[b]{\linewidth}\raggedleft
pvalue.scaled
\end{minipage} & \begin{minipage}[b]{\linewidth}\raggedleft
cfi.scaled
\end{minipage} & \begin{minipage}[b]{\linewidth}\raggedleft
rmsea.scaled
\end{minipage} & \begin{minipage}[b]{\linewidth}\raggedleft
srmr
\end{minipage} \\
\midrule\noalign{}
\endfirsthead
\toprule\noalign{}
\begin{minipage}[b]{\linewidth}\raggedright
Model
\end{minipage} & \begin{minipage}[b]{\linewidth}\raggedleft
chisq.scaled
\end{minipage} & \begin{minipage}[b]{\linewidth}\raggedleft
df.scaled
\end{minipage} & \begin{minipage}[b]{\linewidth}\raggedleft
pvalue.scaled
\end{minipage} & \begin{minipage}[b]{\linewidth}\raggedleft
cfi.scaled
\end{minipage} & \begin{minipage}[b]{\linewidth}\raggedleft
rmsea.scaled
\end{minipage} & \begin{minipage}[b]{\linewidth}\raggedleft
srmr
\end{minipage} \\
\midrule\noalign{}
\endhead
\bottomrule\noalign{}
\endlastfoot
Configural & 64.608 & 48 & 0.055 & 0.950 & 0.070 & 0.089 \\
Metric & 79.964 & 54 & 0.012 & 0.921 & 0.082 & 0.103 \\
Scalar & 78.986 & 63 & 0.084 & 0.952 & 0.060 & 0.092 \\
\end{longtable}

\begin{Shaded}
\begin{Highlighting}[]
\FunctionTok{kable}\NormalTok{(deltas }\SpecialCharTok{\%\textgreater{}\%} \FunctionTok{mutate}\NormalTok{(}\FunctionTok{across}\NormalTok{(}\FunctionTok{where}\NormalTok{(is.numeric), round, }\DecValTok{3}\NormalTok{)), }
      \AttributeTok{caption =} \StringTok{"Delta fit (CFI, RMSEA) thresholds."}\NormalTok{)}
\end{Highlighting}
\end{Shaded}

\begin{longtable}[]{@{}lrr@{}}
\caption{Delta fit (CFI, RMSEA) thresholds.}\tabularnewline
\toprule\noalign{}
step & dCFI & dRMSEA \\
\midrule\noalign{}
\endfirsthead
\toprule\noalign{}
step & dCFI & dRMSEA \\
\midrule\noalign{}
\endhead
\bottomrule\noalign{}
\endlastfoot
Configural -\textgreater{} Metric & -0.028 & 0.012 \\
Metric -\textgreater{} Scalar & 0.030 & -0.022 \\
\end{longtable}

\hypertarget{reproducibility-appendix}{%
\section{Reproducibility appendix}\label{reproducibility-appendix}}

\begin{Shaded}
\begin{Highlighting}[]
\FunctionTok{sessionInfo}\NormalTok{()}
\end{Highlighting}
\end{Shaded}

\begin{verbatim}
## R version 4.4.2 (2024-10-31)
## Platform: x86_64-pc-linux-gnu
## Running under: Ubuntu 24.04.3 LTS
## 
## Matrix products: default
## BLAS:   /usr/lib/x86_64-linux-gnu/blas/libblas.so.3.12.0 
## LAPACK: /usr/lib/x86_64-linux-gnu/lapack/liblapack.so.3.12.0
## 
## locale:
##  [1] LC_CTYPE=en_US.UTF-8       LC_NUMERIC=C               LC_TIME=cs_CZ.UTF-8        LC_COLLATE=en_US.UTF-8    
##  [5] LC_MONETARY=cs_CZ.UTF-8    LC_MESSAGES=en_US.UTF-8    LC_PAPER=cs_CZ.UTF-8       LC_NAME=C                 
##  [9] LC_ADDRESS=C               LC_TELEPHONE=C             LC_MEASUREMENT=cs_CZ.UTF-8 LC_IDENTIFICATION=C       
## 
## time zone: Europe/Prague
## tzcode source: system (glibc)
## 
## attached base packages:
## [1] stats4    stats     graphics  grDevices utils     datasets  methods   base     
## 
## other attached packages:
##  [1] mirt_1.45.1    lattice_0.22-5 car_3.1-3      carData_3.0-5  semTools_0.5-7 lavaan_0.6-20  difR_6.1.0    
##  [8] janitor_2.2.1  stringr_1.5.1  knitr_1.50     psych_2.4.12   ggplot2_4.0.1  tidyr_1.3.1    dplyr_1.1.4   
## 
## loaded via a namespace (and not attached):
##   [1] RColorBrewer_1.1-3   rstudioapi_0.17.1    audio_0.1-11         shape_1.4.6.1        magrittr_2.0.3      
##   [6] TH.data_1.1-4        estimability_1.5.1   farver_2.1.2         nloptr_2.2.1         rmarkdown_2.29      
##  [11] fs_1.6.5             vctrs_0.6.5          minqa_1.2.8          htmltools_0.5.8.1    forcats_1.0.0       
##  [16] haven_2.5.4          cellranger_1.1.0     Formula_1.2-5        dcurver_0.9.3        parallelly_1.45.1   
##  [21] testthat_3.3.1       sandwich_3.1-1       emmeans_1.10.6       rootSolve_1.8.2.4    zoo_1.8-14          
##  [26] lubridate_1.9.4      admisc_0.39          lifecycle_1.0.4      iterators_1.0.14     pkgconfig_2.0.3     
##  [31] Matrix_1.7-1         R6_2.6.1             fastmap_1.2.0        rbibutils_2.3        future_1.68.0       
##  [36] snakecase_0.11.1     digest_0.6.37        Exact_3.3            vegan_2.7-2          progressr_0.18.0    
##  [41] timechange_0.3.0     abind_1.4-8          httr_1.4.7           mgcv_1.9-1           compiler_4.4.2      
##  [46] proxy_0.4-27         withr_3.0.2          S7_0.2.1             R.utils_2.13.0       MASS_7.3-61         
##  [51] sessioninfo_1.2.3    GPArotation_2024.3-1 permute_0.9-8        gld_2.6.7            tools_4.4.2         
##  [56] pbivnorm_0.6.0       future.apply_1.20.0  clipr_0.8.0          R.oo_1.27.1          glue_1.8.0          
##  [61] quadprog_1.5-8       nlme_3.1-166         grid_4.4.2           cluster_2.1.8        generics_0.1.3      
##  [66] gtable_0.3.6         tzdb_0.5.0           R.methodsS3_1.8.2    class_7.3-22         data.table_1.17.8   
##  [71] lmom_3.2             hms_1.1.3            Deriv_4.1.6          foreach_1.5.2        pillar_1.10.0       
##  [76] splines_4.4.2        survival_3.7-0       tidyselect_1.2.1     pbapply_1.7-4        reformulas_0.4.1    
##  [81] gridExtra_2.3        deltaPlotR_1.6       xfun_0.54            expm_1.0-0           brio_1.1.5          
##  [86] stringi_1.8.4        VGAM_1.1-14          yaml_2.3.10          boot_1.3-31          evaluate_1.0.5      
##  [91] codetools_0.2-20     beepr_2.0            msm_1.8.2            tibble_3.2.1         cli_3.6.5           
##  [96] xtable_1.8-4         DescTools_0.99.60    Rdpack_2.6.4         Rcpp_1.0.13-1        readxl_1.4.3        
## [101] globals_0.18.0       polycor_0.8-1        coda_0.19-4.1        parallel_4.4.2       readr_2.1.5         
## [106] lme4_1.1-38          listenv_0.10.0       glmnet_4.1-10        mvtnorm_1.3-2        SimDesign_2.21      
## [111] scales_1.4.0         e1071_1.7-16         purrr_1.1.0          rlang_1.1.6          multcomp_1.4-28     
## [116] mnormt_2.1.1         ltm_1.2-0
\end{verbatim}

\end{document}
