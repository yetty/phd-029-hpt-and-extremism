\documentclass[11pt,a4paper]{article}

% Encoding and language
\usepackage[T1]{fontenc}
\usepackage[utf8]{inputenc}
\usepackage[czech,english]{babel}

% Page layout
\usepackage{geometry}
\geometry{
    a4paper,
    margin=2.5cm
}

% Fonts
\usepackage{lmodern}
\usepackage{inconsolata} % nicer monospaced font for variable names

% Tables
\usepackage{longtable}
\usepackage{booktabs}
\usepackage{array}
\usepackage{multirow}

% Spacing and links
\usepackage{setspace}
\onehalfspacing
\usepackage[hidelinks]{hyperref}

% Paragraph spacing
\setlength{\parskip}{6pt}
\setlength{\parindent}{0pt}

\title{Codebook for \texttt{normalised\_responses}}
\author{Juda Kaleta}
\date{\today}

\begin{document}
\maketitle

\textbf{Data files.} The dataset is distributed in two formats: 
\texttt{normalised\_responses\_<DATE>.RData} (R native) and 
\texttt{normalised\_responses\_<DATE>.xlsx} (Excel). 
Variable names and coding are identical across formats.

\smallskip

\paragraph{Metadata and context}
These variables describe the school, class, and respondent context and support filtering, grouping, and multi-level models.

{\small
\begin{longtable}{p{2cm} p{1cm} p{5.5cm} p{4cm}}
\caption{Codebook --- Metadata}\\
\toprule
\textbf{Variable} & \textbf{Type} & \textbf{Description} & \textbf{Coding}\\
\midrule
\endfirsthead
\toprule
\textbf{Variable} & \textbf{Type} & \textbf{Description} & \textbf{Coding}\\
\midrule
\endhead
\bottomrule
\endfoot

\texttt{school\_id}     & factor  & School identifier (anonymized) & S01, S02, \dots \\
\texttt{school\_level}  & factor  & School level (e.g., vocational, gymnasium, lower/higher secondary\dots) & categorical \\
\texttt{school\_type}   & factor  & School type (public, church, private) & categorical \\
\texttt{region}         & factor  & Czech region of the school & categorical \\
\texttt{class\_label}   & factor  & Class identifier & e.g., 1, 2, 9A, septima \\
\texttt{gender}         & factor  & Student gender (recoded) & M~=~male; F~=~female; O~=~other \\
\texttt{history\_grade} & integer & Last reported history grade & 1--5 \\
\end{longtable}
}

\paragraph{Historical knowledge (KN)}
Items KN1--KN6 are multiple-choice knowledge questions scored via exact-text matching. Each is coded 0 (incorrect) or 1 (correct).
{\small
\begin{longtable}{p{2cm} p{1cm} p{5.5cm} p{4cm}}
\caption{Codebook --- Knowledge}\\
\toprule
\textbf{Variable} & \textbf{Type} & \textbf{Description} & \textbf{Coding}\\
\midrule
\endfirsthead
\toprule
\textbf{Variable} & \textbf{Type} & \textbf{Description} & \textbf{Coding}\\
\midrule
\endhead
\bottomrule
\endfoot

\texttt{KN1-KN6}  & integer & Knowledge items on events of 1918--1929 & 0 = incorrect, 1 = correct \\
\end{longtable}

\paragraph{Historical Perspective-Taking (HPT)}
The HPT instrument assesses how well explanations align with Hannes's perspective in the 1930 vignette. Items are organised into three reasoning modes: POP (populist), ROA (role-of-agent), and CONT (context). Responses use a 1--4 “fit to situation’’ scale.

{\small
\begin{longtable}{p{2cm} p{1cm} p{5.5cm} p{4cm}}
\caption{Codebook --- HPT}\\
\toprule
\textbf{Variable} & \textbf{Type} & \textbf{Description} & \textbf{Coding}\\
\midrule
\endfirsthead
\toprule
\textbf{Variable} & \textbf{Type} & \textbf{Description} & \textbf{Coding}\\
\midrule
\endhead
\bottomrule
\endfoot

\texttt{POP1-POP3}   & integer & HPT populist reasoning items & 1--4 (higher = better fit) \\
\texttt{ROA1-ROA3}   & integer & HPT role-of-agent reasoning items & 1--4 \\
\texttt{CONT1-CONT3} & integer & HPT context reasoning items & 1--4 \\
\end{longtable}
}

\paragraph{FR-LF mini (Right-wing authoritarian attitudes)}
Two short subscales: RD (endorsement of a strict leader / one-party rule) and NS (statements normalising or relativising Nazi crimes). Items use a standard 1--5 Likert scale.

{\small
\begin{longtable}{p{2cm} p{1cm} p{5.5cm} p{4cm}}
\caption{Codebook --- FR-LF mini}\\
\toprule
\textbf{Variable} & \textbf{Type} & \textbf{Description} & \textbf{Coding}\\
\midrule
\endfirsthead
\toprule
\textbf{Variable} & \textbf{Type} & \textbf{Description} & \textbf{Coding}\\
\midrule
\endhead
\bottomrule
\endfoot

\texttt{RD1-RD3} & integer & Right-wing dictatorship acceptance & 1--5 (Likert) \\
\texttt{NS1-NS3} & integer & Nazi-sympathy / relativization statements & 1--5 (Likert) \\
\end{longtable}
}

\paragraph{KSA-3 (Authoritarianism)}
An established 9-item authoritarianism short scale with three three-item facets: A (authoritarian aggression), U (submission/obedience), and K (conventionalism). All items use a 1--5 Likert response.

{\small
\begin{longtable}{p{2cm} p{1cm} p{5.5cm} p{4cm}}
\caption{Codebook --- KSA-3}\\
\toprule
\textbf{Variable} & \textbf{Type} & \textbf{Description} & \textbf{Coding}\\
\midrule
\endfirsthead
\toprule
\textbf{Variable} & \textbf{Type} & \textbf{Description} & \textbf{Coding}\\
\midrule
\endhead
\bottomrule
\endfoot

\texttt{A1-A3} & integer & Authoritarian aggression subscale & 1--5 (Likert) \\
\texttt{U1-U3} & integer & Submission / obedience subscale & 1--5 (Likert) \\
\texttt{K1-K3} & integer & Conventionalism subscale & 1--5 (Likert) \\
\end{longtable}
}

\paragraph{SDR-5 (Social desirability)}
A brief social desirability scale. SDR2--SDR4 are already reversed in the dataset (1$\leftrightarrow$5). Items can be summed or averaged depending on analytic needs.

{\small
\begin{longtable}{p{2cm} p{1cm} p{5.5cm} p{4cm}}
\caption{Codebook --- SDR-5}\\
\toprule
\textbf{Variable} & \textbf{Type} & \textbf{Description} & \textbf{Coding}\\
\midrule
\endfirsthead
\toprule
\textbf{Variable} & \textbf{Type} & \textbf{Description} & \textbf{Coding}\\
\midrule
\endhead
\bottomrule
\endfoot

\texttt{SDR1} & integer & Social desirability item 1 & 1--5 (Likert) \\
\texttt{SDR2} & integer & Social desirability item 2 (reversed) & 1--5 (reversed Likert) \\
\texttt{SDR3} & integer & Social desirability item 3 (reversed) & 1--5 (reversed Likert) \\
\texttt{SDR4} & integer & Social desirability item 4 (reversed) & 1--5 (reversed Likert) \\
\texttt{SDR5} & integer & Social desirability item 5 & 1--5 (Likert) \\
\end{longtable}
}

\end{document}
